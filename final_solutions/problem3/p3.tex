\documentclass[11pt]{article}
\usepackage{graphicx}    % needed for including graphics e.g. EPS, PS
\usepackage{epstopdf}
\usepackage{amsmath}
\usepackage{hyperref}
\usepackage{xspace}
\usepackage{mathtools}
\usepackage{tikz}
\usepackage{epsfig}
\usepackage{float}
%\usepackage{natbib}
\usepackage{subfigure}
\usepackage{setspace}
\usepackage{tabularx,ragged2e,booktabs,caption}


\setlength{\oddsidemargin}{0.1in}
\setlength{\textwidth}{7.25in}

\setlength{\topmargin}{-1in}     %\topmargin: gap above header
\setlength{\headheight}{0in}     %\headheight: height of header
\setlength{\topskip}{0in}        %\topskip: between header and text
\setlength{\headsep}{0in}        
\setlength{\textheight}{692pt}   %\textheight: height of main text
\setlength{\textwidth}{7.5in}    % \textwidth: width of text
\setlength{\oddsidemargin}{-0.5in}  % \oddsidemargin: odd page left margin
\setlength{\evensidemargin}{0in} %\evensidemargin : even page left margin
\setlength{\parindent}{0.25in}   %\parindent: indentation of paragraphs
\setlength{\parskip}{0pt}        %\parskip: gap between paragraphs
\setlength{\voffset}{0.5in}

\newcommand{\DOT}{\text{\textbullet}}
\newcommand{\B}{\mathbf{B}}
\newcommand{\U}{\mathbf{u}}
\newcommand{\PD}{\partial}
\newcommand{\E}{\mathbf{\hat{e}}}

% Useful commands:

% \hfill		aligns-right everything right of \hfill

\begin{document}
\doublespacing
\title{Problem 3}
\author{C. Kawczynski \\
Department of Mechanical and Aerospace Engineering \\
University of California Los Angeles, USA\\
}
\maketitle

\section{Circular pipe with insulating walls}
\subsection{Question}
Consider a fully developed MHD flow in a non-conducting circular pipe with radius $R$ in the presence of a uniform magnetic field in the z-direction ($\B = B \E_z$) as shown in the figure.

For such a configuration evaluate the following:

a) The distribution of electric potential along the wall ($r=R$) of the pipe for a given axisymmetric velocity profile $\U(r) = 2 u_{ave} \left( 1 - \frac{r^2}{R^2} \right) \E_x$ (here $u_{ave}$ is the average fluid velocity) by solving a 2D Poisson equation for the electric potential in the y-z plane with the assumption that the velocity profile is not affected by the magnetic field. [HINT: use the method of separation of variables.]

b) Potential difference between points A and B for the magnetic field strength B of 1 Tesla, average velocity $u_{ave}$ of 10 cm/sec and pipe radius $R$ of 10 cm.

\subsection{Solution}
Assuming axi-symmetric flow and
\begin{equation}
	\B = B \E_z
	, \qquad
	\U = \U(r) \E_x.
\end{equation}

The governing equation for $\phi$ is
\begin{equation}
	\nabla^2 \phi = \B \DOT \omega
\end{equation}
with
\begin{equation}
	\frac{\PD \phi}{\PD r} = 0, \qquad \text{for} \qquad r=R.
\end{equation}
Evaluating this we have
\begin{equation}
	\nabla^2\phi 
	= -B \frac{\PD u}{\PD y}
	= -B \frac{\PD u}{\PD r} \frac{\PD r}{\PD y}
	= -B \frac{\PD u}{\PD r} \cos(\theta)
\end{equation}
Using the vector identity for the Laplacian operator in cylindrical coordinates, we have
\begin{equation}
	\nabla^2 \phi = 
	\frac{1}{r} \frac{\PD}{\PD r} \left( r \frac{\PD \phi}{\PD r} \right)
	+ \frac{1}{r^2} \frac{\PD^2 \phi}{\PD \theta^2}
\end{equation}
Equating the last two equations, we have
\begin{equation}
	\frac{1}{r} \frac{\PD}{\PD r} \left( r \frac{\PD \phi}{\PD r} \right)
	+ \frac{1}{r^2} \frac{\PD^2 \phi}{\PD \theta^2}
	=
	-B \frac{\PD u}{\PD r} \cos(\theta)
\end{equation}
Use the ansatz $\phi(r,\theta) = \cos(\theta) f(r)$ to get
\begin{equation}
	\frac{d}{dr} \left( r^2 \frac{df}{dr} - rf \right)
	=
	-r^2 Bu'
\end{equation}
Apply the BC $\frac{df}{dr}=0$ at $r=R$ and integrating the above equation from 0 to $R$ to get
\begin{equation}
	\left[ r^2 \frac{df}{dr} - rf \right]_0^R
	=
	-B \int_0^R r^2 u' dr
	= \left[ -Br^2 u \right]_0^R + 2B \int_0^R r-u dr
\end{equation}

Second term on RHS of above equation is the flow rate and the first term goes to zero.
\begin{equation}
	R f(R) = \frac{2B \Phi}{2 \pi}
\end{equation}
Where
\begin{equation}
	\Phi = \text{flow rate} = \iint_S \U \DOT ds
	= 2\pi \int_0^R u(r) r dr
	, \qquad \text{(axisymmetric)}
\end{equation}
Therefore we have
\begin{equation}
	\boxed{
	f(R) = \frac{B\Phi}{\pi R}
	}
\end{equation}
Finally
\begin{equation}
	\boxed{
	\phi(R) = \frac{B\Phi}{\pi R} \cos(\theta)
	}
\end{equation}

Voltage difference at points A and B are
\begin{equation}
	\cos(\theta) = -1 \text{ and } +1
\end{equation}
So we have
\begin{equation}
	|\phi_A - \phi_B| 
	= \phi(R,\pi) - \phi(R,0)
	=-2f(R)
\end{equation}
Therefore
\begin{equation}
	\boxed{|\phi_A - \phi_B | = \frac{2B\Phi}{\pi R}}
\end{equation}
This shows that the solution is independent of the velocity profile provided it is axisymmetric.

\end{document}