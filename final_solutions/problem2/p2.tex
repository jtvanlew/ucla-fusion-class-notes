\documentclass[11pt]{article}
\usepackage{graphicx}    % needed for including graphics e.g. EPS, PS
\usepackage{epstopdf}
\usepackage{amsmath}
\usepackage{hyperref}
\usepackage{xspace}
\usepackage{mathtools}
\usepackage{tikz}
\usepackage{epsfig}
\usepackage{float}
%\usepackage{natbib}
\usepackage{subfigure}
\usepackage{setspace}
\usepackage{tabularx,ragged2e,booktabs,caption}


\setlength{\oddsidemargin}{0.1in}
\setlength{\textwidth}{7.25in}

\setlength{\topmargin}{-1in}     %\topmargin: gap above header
\setlength{\headheight}{0in}     %\headheight: height of header
\setlength{\topskip}{0in}        %\topskip: between header and text
\setlength{\headsep}{0in}        
\setlength{\textheight}{692pt}   %\textheight: height of main text
\setlength{\textwidth}{7.5in}    % \textwidth: width of text
\setlength{\oddsidemargin}{-0.5in}  % \oddsidemargin: odd page left margin
\setlength{\evensidemargin}{0in} %\evensidemargin : even page left margin
\setlength{\parindent}{0.25in}   %\parindent: indentation of paragraphs
\setlength{\parskip}{0pt}        %\parskip: gap between paragraphs
\setlength{\voffset}{0.5in}

\newcommand{\DOT}{\text{\textbullet}}
\newcommand{\B}{\mathbf{B}}
\newcommand{\A}{\mathbf{A}}
\newcommand{\J}{\mathbf{j}}
\newcommand{\U}{\mathbf{u}}
\newcommand{\PD}{\partial}
\newcommand{\E}{\mathbf{\hat{e}}}

% Useful commands:

% \hfill		aligns-right everything right of \hfill

\begin{document}
\doublespacing
\title{Problem 2}
\author{C. Kawczynski \\
Department of Mechanical and Aerospace Engineering \\
University of California Los Angeles, USA\\
}
\maketitle

\section{Derive the vorticity equation with the Lorentz force}
\subsection{Question}
Derive the vorticity equation $\omega = \frac{\PD U}{\PD y} - \frac{\PD V}{\PD x}$ for a 2D MHD flow (in the x-y plane) of electrically conducting fluid in a constant spanwise magnetic field (the field is in z-direction). Based on this equation, conclude what kind of MHD effect will be experienced by the flow.

\subsection{Solution}
\underline{Assumptions}

First, not that $\J = \nabla \times \left( \frac{\B}{\mu}\right)$ and $\PD_z () = 0$, therefore $\J = j \mathbf{\hat{e}}_z$. So our assumptions are
\begin{enumerate}
\item two-dimensional ($\PD_z() = 0$)
\item 1 component of vorticity $\omega_x = \omega_y = 0, \mathbf{\omega} = \omega_z$
\item induced magnetic field is small compared to applied (low magnetic Reynolds number $Re_m << 1$)
\item currents close at infinity $(\J = j \mathbf{\hat{e}}_z)$
\end{enumerate}

\underline{Analysis}

The vorticity equation is derived by taking the curl of the momentum equation.
\begin{equation*}
	\epsilon_{lmi} \PD_m
	\left( 
	\underbrace{\PD_t u_i}_{1} +
	\underbrace{u_j \PD_j u_i}_{2} = 
	\underbrace{-\frac{1}{\rho} \PD_i p}_{3} + 
	\underbrace{\nu \PD_{jj} u_i}_{4} + 
	\underbrace{\frac{1}{\rho} \epsilon_{ijk} j_j B_k}_{5}
	\right)
\end{equation*}
First, let $\omega_l = \epsilon_{lmi} \PD_m u_i$, and we get the following:

\noindent
Unsteady term
\begin{align}
	\epsilon_{lmi} \PD_m \PD_t u_i
	& = \PD_t \epsilon_{lmi} \PD_m u_i \notag \\
	& = \PD_t \omega_l
\end{align}
Convection term
\begin{align}
	\epsilon_{lmi} \PD_m u_j \PD_j u_i
	& = \epsilon_{lmi} \PD_m u_j \PD_j u_i \notag \\
	& = \epsilon_{lmi} \PD_m \PD_j (u_i u_j) \qquad \text{($\PD_i u_i = 0$)} \notag \\
	& = \epsilon_{lmi} \PD_j \PD_m (u_i u_j) \qquad \text{(swap $\PD$ order)} \notag \\
	& = \PD_j \PD_m (\epsilon_{lmi} u_i u_j) \notag \\
	& = \PD_j (\omega_l u_j) \notag \\
	& = u_j \PD_j \omega_l \qquad \text{($\PD_j u_j = 0$)} \notag \\
\end{align}
Pressure term
\begin{align}
	- \epsilon_{lmi} \PD_m \frac{1}{\rho} \PD_i p
	& = -  \frac{1}{\rho} \epsilon_{lmi} \PD_m \PD_i p \notag \\
	& = 0 \qquad \text{(by identity)}
\end{align}
Diffusion term
\begin{align}
	\epsilon_{lmi} \PD_m \nu \PD_{jj} u_i
	& = \nu \PD_m \PD_{jj} \epsilon_{lmi} u_i \notag \\
	& = \nu \PD_j \PD_{mj} \epsilon_{lmi} u_i \qquad \text{(swap $\PD$ order)} \notag \\
	& = \nu \PD_j \PD_{jm} \epsilon_{lmi} u_i \qquad \text{(swap index)} \notag \\
	& = \nu \PD_j \PD_{j} \omega_l \notag \\
	& = \nu \PD_{jj} \omega_l \notag \\
\end{align}
\newpage
\noindent
The last term (5) is the contribution from the electromagnetic Lorentz force. 
\subsubsection{Assuming low magnetic Reynolds number}
Making use of the vector identity
\begin{equation*}
	\nabla \times (\A \times \B) = \A (\nabla \DOT \B) - \B (\nabla \DOT \A) + (\B \DOT \nabla) \A - (\A \DOT \nabla) \B
\end{equation*}
We have
\begin{align}
	\nabla \times \J \times \B
	& = \underbrace{\J (\nabla \DOT \B)}_{=0} - 
	    \underbrace{\B (\nabla \DOT \J)}_{=0} + 
	    (\B \DOT \nabla) \J - 
	    (\J \DOT \nabla) \B \notag \qquad \text{(vector identity)} \\
	& = (\B \DOT \nabla) \J - (\J \DOT \nabla) \B \notag \\
	& = \underbrace{B_x \PD_x \J + B_y \PD_y \J}_{=0,\text{ (assumption 3)}} + \underbrace{B_z \PD_z \J}_{=0, \text{($\PD_z()=0$)}}
	- \left( \underbrace{j_x \PD_x \B + j_y \PD_y \B}_{=0, \text{ (assumption 4)}} + \underbrace{j_z \PD_z \B}_{=0, \text{ ($\PD_z()=0$)}} \right) \notag \\
	& = \mathbf{0}
\end{align}

\subsubsection{Assuming **finite magnetic Reynolds number**}
Here, we relax the assumption of low magnetic Reynolds number. Making use of the vector identity
\begin{equation*}
	\nabla \times (\A \times \B) = \A (\nabla \DOT \B) - \B (\nabla \DOT \A) + (\B \DOT \nabla) \A - (\A \DOT \nabla) \B
\end{equation*}
We have
\begin{align}
	\nabla \times \J \times \B
	& = \underbrace{\J (\nabla \DOT \B)}_{=0} - 
	    \underbrace{\B (\nabla \DOT \J)}_{=0} + 
	    (\B \DOT \nabla) \J - 
	    (\J \DOT \nabla) \B \notag \qquad \text{(vector identity)} \\
	& = (\B \DOT \nabla) \J - (\J \DOT \nabla) \B \notag \\
	& = B_x \PD_x \J + B_y \PD_y \J + \underbrace{B_z \PD_z \J}_{=0, \text{($\PD_z()=0$)}}
	- \left( \underbrace{j_x \PD_x \B + j_y \PD_y \B}_{=0, \text{ (assumption 4)}} + \underbrace{j_z \PD_z \B}_{=0, \text{ ($\PD_z()=0$)}} \right) \notag \\
	& = B_x \PD_x j + B_y \PD_y j
\end{align}

\subsection{Final result}
\subsubsection{Low magnetic Reynolds number}
Putting this all together, we have
\begin{equation*}
	\frac{\PD \omega}{\PD t} +
	(\U \DOT \nabla) \omega =
	\nu \nabla^2 \omega
\end{equation*}
Therefore, the vorticity equation is unaffected by the Lorentz force for this 2D flow.

\subsubsection{Finite magnetic Reynolds number}
Putting this all together, we have
\begin{equation*}
	\frac{\PD \omega}{\PD t} +
	(\U \DOT \nabla) \omega =
	\nu \nabla^2 \omega + (\B \DOT \nabla) \J
\end{equation*}

\end{document}